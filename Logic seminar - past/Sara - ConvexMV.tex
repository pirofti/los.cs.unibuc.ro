\documentclass[10pt]{amsproc}
\usepackage{amssymb,amsmath, amsthm, amsfonts}
\usepackage{mathrsfs}
\usepackage[all]{xypic}

\newcommand{\luk}{\L u\-ka\-si\-e\-w\-icz}
\hyphenation{ho-mo-mor-phism}

\newtheorem{theorem}{Theorem}[section]
\newtheorem{lemma}[theorem]{Lemma}
\newtheorem{corollary}[theorem]{Corollary}
\newtheorem{proposition}[theorem]{Proposition}

\theoremstyle{definition}
\newtheorem{definition}[theorem]{Definition}
\newtheorem{remark}[theorem]{Remark}
\newtheorem{example}[theorem]{Example}
\newcommand{\msim}{\mathord\sim}
\newcommand{\lex}{\times_{lex}}
%\newenvironment{proof}
%{\begin{trivlist} \item[] {\bf Proof. }}
%{\qed\end{trivlist}}


%%%%%end macros personali%%%%%%
\begin{document}

\title{On MV-algebras with convexity operators}
\author{Serafina Lapenta\\ joint work with Tommaso Flaminio}
\address{Department of Mathematics, University of Salerno. Via Ponte Don Melillo - 84084 Fisciano SA - Italy. \texttt{slapenta@unisa.it}}
\date{}

\maketitle


The notions of {\em convexity} plays a central r\^{o}le in logic and mathematics. Starting from a seminal idea of Brown \cite{Br}, we propose an axiomatic approach to convex combinations in the realm of MV-algebras \cite{CDM}. 
%These structure are the equivalent algebraic sematics of the infinite-valued \luk\ calculus, they are categorically equivalent to abelian lattice-ordered groups with strong unit, they form a variety which is generated by the so called {\em standard} MV-algebra: the real unit interval $[0,1]$ endowed with operations $x\oplus y=\min\{0,x+y\}$, $\neg x=1-x$, and the constant $0$. 
More in detail, we will expand the language of MV-algebras by an uncountable family of binary operations $cc_\alpha(\cdot, \cdot)$ (one for every $\alpha\in [0,1]$) axiomatized so to capture the basic properties of convex combinations in $[0,1]$. The so resulting algebras are called {\em convex} MV-algebras (or CMV-algebras for short). 

CMV-algebras form a variety. Our first result shows that CMV-algebras are termwise equivalent to {\em Riesz} MV-algebras \cite{DiNola-Leustean:RMV} and, consequently, the variety of CMV-algebras is generated by the standard CMV-algebra, that is the standard MV-algebra where the operators $cc_\alpha$ are interpreted in the usual way: for each $x,y,\alpha\in [0,1]$, $cc_\alpha(x,y)$ is $\alpha x+(1-\alpha)y$.   

States of MV-algebras \cite{munstates} are analogous to finitely additive probabilities on boolean algebras and, for every MV-algebra ${\bf A}$, its states form a subset of $[0,1]^A$ which coincide with the topological closure of the convex hull of the MV-homomorphisms of ${\bf A}$ in the standard MV-algebra $[0,1]_{MV}$. Thanks to this characterization of the states space, we will show that each state of a finitely dimensional MV-algebra $[0,1]^X$ (with $X$ finite) has a faithful representation in the free CMV-algebra $|X|$-generated. 
\begin{thebibliography}{9}

\bibitem{Br}
N. P. Brown, Topological Dynamical Systems Associated to $\Pi_1$-factors, preprint arXiv:1010.1214.

\bibitem{CDM}
R. Cignoli, I. M. L. D'Ottaviano, D. Mundici, Algebraic Foundations of Many-valued Reasoning, {\em Trends in Logic Vol 8, Kluwer, Dordrecht}, 2000.

\bibitem{DiNola-Leustean:RMV}
A. Di Nola, I. Leu\c stean, \textit{\L ukasiewicz logic and Riesz Spaces}, Soft Computing, Soft Comp. 18(12) (2014) 2349-2363. arXiv:1309.1575v1

\bibitem{munstates}
D. Mundici, Averaging the Truth-value in {\L}ukasiewicz Logic. {\em Studia Logica} 55(1), 113--127, 1995. 

\end{thebibliography}




\end{document}
