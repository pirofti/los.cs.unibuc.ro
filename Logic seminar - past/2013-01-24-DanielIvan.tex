\documentclass[a4paper,12pt]{article}
\usepackage[utf8]{inputenc}
\usepackage{amsmath, amssymb,amscd}
\usepackage{amsthm}
\usepackage{fullpage}

%opening
\title{Logic Seminar: Proof Interpretations IV}
\author{Speaker: Daniel Ivan, IMAR}
\date{January 24, 2013}
\begin{document}

\maketitle
In this seminar we will conclude our study of \emph{monotone modified realizability}, with the following:
\thispagestyle{empty}
\begin{enumerate}
\item Corollary of the Soundness theorem for monotone modified realizability: in ${\bf E-HA}^\omega + {\rm AC} + {\rm IP}_\lnot^\omega + {\rm CA}_\lnot^\omega$, from a proof of 
\begin{equation*}
 \forall x^1 \forall y\leq_\rho sx \exists z^\tau A(x,y,z)
\end{equation*}
we can extract a term $t$ such that we can prove 
\begin{equation*}
\forall x^1 \forall y\leq_\rho sx \exists z\leq_\tau tx A(x,y,z)
\end{equation*}
for any formula $A$, closed term $s$ and types $\rho,\tau$ with $\deg(\tau)=2$.
  \item \emph{Main theorem on program extraction by \emph{monotone} modified realizability}: in ${\bf E-HA}^\omega+{\rm AC}+{\rm IP}^\omega_{\lnot} + \Omega$, from a proof of 
\begin{equation*}
 \forall x^1 \forall y\leq_{\rho} sx \exists z^\tau A(x,y,z)
\end{equation*}
we can extract a term $t$ such that we can prove 
\begin{equation*}
\forall x^1 \forall y\leq_\rho sx \exists z^\tau  A(x,y, tx)
\end{equation*}
for any formula $A(x,y,z)$ in $\mathcal{L}({\bf E-HA}^\omega)$, where $\Omega$ is a set of sentences of the type 
$\forall \underline{u}^{\underline{\delta}} (C(\underline{u}) \rightarrow \exists \underline{v} \leq_{\underline{\sigma}} \underline{r}\,\underline{u} \lnot B(\underline{u}, \underline{v}))$, with $C$ and $B$ arbitrary formulas of ${\bf E-HA}^\omega$, and $\rho$, $\tau$, $\underline{\sigma}$, $\underline{\delta}$ are arbitrary (tuples of) types.
\item Applications of (monotone) modified realizability:
\begin{itemize}
 \item If $\Phi_{(\cdot)}^{1(1)(0)}$, $\Phi^{1(1)}$ are closed terms of ${\bf E-HA}^\omega$ satisfying
\begin{equation*}
 \forall x, y\leq_1 N \forall n^0 (\tilde{x} =_\mathbb{R} \tilde{y} \rightarrow \Phi_n(\tilde{x}) =_\mathbb{R} \Phi_n(\tilde{y}) \land \Phi(\tilde{x}) =_\mathbb{R} \Phi(\tilde{y}))
\end{equation*}
then provable pointwise convergence of $\Phi_n$ towards $\Phi$ on $[0,1]$ implies provable uniform convergence on $[0,1]$.

\item The \emph{Weak Markov Principle}  (WMP for short), is not provable in ${\bf E-HA}^\omega + {\rm AC} + {\rm CA}_\lnot^\omega$, where WMP:``\emph{Every pseudo-positive real number is positive}'', and $a\in\mathbb{R}$ is pseudo-positive  if 
$\forall x\in \mathbb{R} (\lnot\lnot (0<x) \lor \lnot \lnot (x<a))$.

\end{itemize}

\end{enumerate}
\end{document}
