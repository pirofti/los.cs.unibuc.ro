\documentclass[12pt]{article}
\newcommand{\n}{\noindent}
\usepackage{amssymb}


\begin{document}

\begin{center}
Jacek Malinowski\\
Polish Academy of Sciences\\

\vskip1cm

{\Large \textbf{Theory of Logical Consequence\\
- past, present and future}}
\end{center}

\vskip2cm

\begin{center}
{\bf (abstract)}
\end{center}

\n The aim of the lecture is to give a general review of old and
more recent results, as well as, possible future developments in
the theory of logical consequence operation.

\begin{itemize}
  \item I start from comparing three frameworks for logical
investigations: Tarski's logical consequence operation, Gentzen's
sequents and Hilbert-style proofs.
  \item In the second part I will give a review of results in the
theory of logical consequence operation concerning logical
matrices, algebraic semantics, implicative logics and
equivalential logic.
  \item Last part concerns some generalizations of logical
consequence and logical matrices. The former gives us an interesting and promising conceptual framework for non-monotonic logic. The later gives an general semantics for logic of rejection.    
\end{itemize}


\end{document}
